
% Default to the notebook output style

    


% Inherit from the specified cell style.




    
\documentclass[11pt]{article}

    
    
    \usepackage[T1]{fontenc}
    % Nicer default font (+ math font) than Computer Modern for most use cases
    \usepackage{mathpazo}

    % Basic figure setup, for now with no caption control since it's done
    % automatically by Pandoc (which extracts ![](path) syntax from Markdown).
    \usepackage{graphicx}
    % We will generate all images so they have a width \maxwidth. This means
    % that they will get their normal width if they fit onto the page, but
    % are scaled down if they would overflow the margins.
    \makeatletter
    \def\maxwidth{\ifdim\Gin@nat@width>\linewidth\linewidth
    \else\Gin@nat@width\fi}
    \makeatother
    \let\Oldincludegraphics\includegraphics
    % Set max figure width to be 80% of text width, for now hardcoded.
    \renewcommand{\includegraphics}[1]{\Oldincludegraphics[width=.8\maxwidth]{#1}}
    % Ensure that by default, figures have no caption (until we provide a
    % proper Figure object with a Caption API and a way to capture that
    % in the conversion process - todo).
    \usepackage{caption}
    \DeclareCaptionLabelFormat{nolabel}{}
    \captionsetup{labelformat=nolabel}

    \usepackage{adjustbox} % Used to constrain images to a maximum size 
    \usepackage{xcolor} % Allow colors to be defined
    \usepackage{enumerate} % Needed for markdown enumerations to work
    \usepackage{geometry} % Used to adjust the document margins
    \usepackage{amsmath} % Equations
    \usepackage{amssymb} % Equations
    \usepackage{textcomp} % defines textquotesingle
    % Hack from http://tex.stackexchange.com/a/47451/13684:
    \AtBeginDocument{%
        \def\PYZsq{\textquotesingle}% Upright quotes in Pygmentized code
    }
    \usepackage{upquote} % Upright quotes for verbatim code
    \usepackage{eurosym} % defines \euro
    \usepackage[mathletters]{ucs} % Extended unicode (utf-8) support
    \usepackage[utf8x]{inputenc} % Allow utf-8 characters in the tex document
    \usepackage{fancyvrb} % verbatim replacement that allows latex
    \usepackage{grffile} % extends the file name processing of package graphics 
                         % to support a larger range 
    % The hyperref package gives us a pdf with properly built
    % internal navigation ('pdf bookmarks' for the table of contents,
    % internal cross-reference links, web links for URLs, etc.)
    \usepackage{hyperref}
    \usepackage{longtable} % longtable support required by pandoc >1.10
    \usepackage{booktabs}  % table support for pandoc > 1.12.2
    \usepackage[inline]{enumitem} % IRkernel/repr support (it uses the enumerate* environment)
    \usepackage[normalem]{ulem} % ulem is needed to support strikethroughs (\sout)
                                % normalem makes italics be italics, not underlines
    \usepackage{mathrsfs}
\usepackage{verbatim}
    \usepackage{float}
    \restylefloat{table}

    
    
    % Colors for the hyperref package
    \definecolor{urlcolor}{rgb}{0,.145,.698}
    \definecolor{linkcolor}{rgb}{.71,0.21,0.01}
    \definecolor{citecolor}{rgb}{.12,.54,.11}

    % ANSI colors
    \definecolor{ansi-black}{HTML}{3E424D}
    \definecolor{ansi-black-intense}{HTML}{282C36}
    \definecolor{ansi-red}{HTML}{E75C58}
    \definecolor{ansi-red-intense}{HTML}{B22B31}
    \definecolor{ansi-green}{HTML}{00A250}
    \definecolor{ansi-green-intense}{HTML}{007427}
    \definecolor{ansi-yellow}{HTML}{DDB62B}
    \definecolor{ansi-yellow-intense}{HTML}{B27D12}
    \definecolor{ansi-blue}{HTML}{208FFB}
    \definecolor{ansi-blue-intense}{HTML}{0065CA}
    \definecolor{ansi-magenta}{HTML}{D160C4}
    \definecolor{ansi-magenta-intense}{HTML}{A03196}
    \definecolor{ansi-cyan}{HTML}{60C6C8}
    \definecolor{ansi-cyan-intense}{HTML}{258F8F}
    \definecolor{ansi-white}{HTML}{C5C1B4}
    \definecolor{ansi-white-intense}{HTML}{A1A6B2}
    \definecolor{ansi-default-inverse-fg}{HTML}{FFFFFF}
    \definecolor{ansi-default-inverse-bg}{HTML}{000000}

    % commands and environments needed by pandoc snippets
    % extracted from the output of `pandoc -s`
    \providecommand{\tightlist}{%
      \setlength{\itemsep}{0pt}\setlength{\parskip}{0pt}}
    \DefineVerbatimEnvironment{Highlighting}{Verbatim}{commandchars=\\\{\}}
    % Add ',fontsize=\small' for more characters per line
    \newenvironment{Shaded}{}{}
    \newcommand{\KeywordTok}[1]{\textcolor[rgb]{0.00,0.44,0.13}{\textbf{{#1}}}}
    \newcommand{\DataTypeTok}[1]{\textcolor[rgb]{0.56,0.13,0.00}{{#1}}}
    \newcommand{\DecValTok}[1]{\textcolor[rgb]{0.25,0.63,0.44}{{#1}}}
    \newcommand{\BaseNTok}[1]{\textcolor[rgb]{0.25,0.63,0.44}{{#1}}}
    \newcommand{\FloatTok}[1]{\textcolor[rgb]{0.25,0.63,0.44}{{#1}}}
    \newcommand{\CharTok}[1]{\textcolor[rgb]{0.25,0.44,0.63}{{#1}}}
    \newcommand{\StringTok}[1]{\textcolor[rgb]{0.25,0.44,0.63}{{#1}}}
    \newcommand{\CommentTok}[1]{\textcolor[rgb]{0.38,0.63,0.69}{\textit{{#1}}}}
    \newcommand{\OtherTok}[1]{\textcolor[rgb]{0.00,0.44,0.13}{{#1}}}
    \newcommand{\AlertTok}[1]{\textcolor[rgb]{1.00,0.00,0.00}{\textbf{{#1}}}}
    \newcommand{\FunctionTok}[1]{\textcolor[rgb]{0.02,0.16,0.49}{{#1}}}
    \newcommand{\RegionMarkerTok}[1]{{#1}}
    \newcommand{\ErrorTok}[1]{\textcolor[rgb]{1.00,0.00,0.00}{\textbf{{#1}}}}
    \newcommand{\NormalTok}[1]{{#1}}
    
    % Additional commands for more recent versions of Pandoc
    \newcommand{\ConstantTok}[1]{\textcolor[rgb]{0.53,0.00,0.00}{{#1}}}
    \newcommand{\SpecialCharTok}[1]{\textcolor[rgb]{0.25,0.44,0.63}{{#1}}}
    \newcommand{\VerbatimStringTok}[1]{\textcolor[rgb]{0.25,0.44,0.63}{{#1}}}
    \newcommand{\SpecialStringTok}[1]{\textcolor[rgb]{0.73,0.40,0.53}{{#1}}}
    \newcommand{\ImportTok}[1]{{#1}}
    \newcommand{\DocumentationTok}[1]{\textcolor[rgb]{0.73,0.13,0.13}{\textit{{#1}}}}
    \newcommand{\AnnotationTok}[1]{\textcolor[rgb]{0.38,0.63,0.69}{\textbf{\textit{{#1}}}}}
    \newcommand{\CommentVarTok}[1]{\textcolor[rgb]{0.38,0.63,0.69}{\textbf{\textit{{#1}}}}}
    \newcommand{\VariableTok}[1]{\textcolor[rgb]{0.10,0.09,0.49}{{#1}}}
    \newcommand{\ControlFlowTok}[1]{\textcolor[rgb]{0.00,0.44,0.13}{\textbf{{#1}}}}
    \newcommand{\OperatorTok}[1]{\textcolor[rgb]{0.40,0.40,0.40}{{#1}}}
    \newcommand{\BuiltInTok}[1]{{#1}}
    \newcommand{\ExtensionTok}[1]{{#1}}
    \newcommand{\PreprocessorTok}[1]{\textcolor[rgb]{0.74,0.48,0.00}{{#1}}}
    \newcommand{\AttributeTok}[1]{\textcolor[rgb]{0.49,0.56,0.16}{{#1}}}
    \newcommand{\InformationTok}[1]{\textcolor[rgb]{0.38,0.63,0.69}{\textbf{\textit{{#1}}}}}
    \newcommand{\WarningTok}[1]{\textcolor[rgb]{0.38,0.63,0.69}{\textbf{\textit{{#1}}}}}
    
    
    % Define a nice break command that doesn't care if a line doesn't already
    % exist.
    \def\br{\hspace*{\fill} \\* }
    % Math Jax compatibility definitions
    \def\gt{>}
    \def\lt{<}
    \let\Oldtex\TeX
    \let\Oldlatex\LaTeX
    \renewcommand{\TeX}{\textrm{\Oldtex}}
    \renewcommand{\LaTeX}{\textrm{\Oldlatex}}
    % Document parameters
    % Document title
   \title{Prétraitement - Importation et nettoyage de données}
    \author{
	Amina, KACIMI\\
	\and
	Valentin, DEHAINAULT\\
	\and
	Minh-Hoang, DANG\\	
}   
    
    
    
    

    % Pygments definitions
    
\makeatletter
\def\PY@reset{\let\PY@it=\relax \let\PY@bf=\relax%
    \let\PY@ul=\relax \let\PY@tc=\relax%
    \let\PY@bc=\relax \let\PY@ff=\relax}
\def\PY@tok#1{\csname PY@tok@#1\endcsname}
\def\PY@toks#1+{\ifx\relax#1\empty\else%
    \PY@tok{#1}\expandafter\PY@toks\fi}
\def\PY@do#1{\PY@bc{\PY@tc{\PY@ul{%
    \PY@it{\PY@bf{\PY@ff{#1}}}}}}}
\def\PY#1#2{\PY@reset\PY@toks#1+\relax+\PY@do{#2}}

\expandafter\def\csname PY@tok@w\endcsname{\def\PY@tc##1{\textcolor[rgb]{0.73,0.73,0.73}{##1}}}
\expandafter\def\csname PY@tok@c\endcsname{\let\PY@it=\textit\def\PY@tc##1{\textcolor[rgb]{0.25,0.50,0.50}{##1}}}
\expandafter\def\csname PY@tok@cp\endcsname{\def\PY@tc##1{\textcolor[rgb]{0.74,0.48,0.00}{##1}}}
\expandafter\def\csname PY@tok@k\endcsname{\let\PY@bf=\textbf\def\PY@tc##1{\textcolor[rgb]{0.00,0.50,0.00}{##1}}}
\expandafter\def\csname PY@tok@kp\endcsname{\def\PY@tc##1{\textcolor[rgb]{0.00,0.50,0.00}{##1}}}
\expandafter\def\csname PY@tok@kt\endcsname{\def\PY@tc##1{\textcolor[rgb]{0.69,0.00,0.25}{##1}}}
\expandafter\def\csname PY@tok@o\endcsname{\def\PY@tc##1{\textcolor[rgb]{0.40,0.40,0.40}{##1}}}
\expandafter\def\csname PY@tok@ow\endcsname{\let\PY@bf=\textbf\def\PY@tc##1{\textcolor[rgb]{0.67,0.13,1.00}{##1}}}
\expandafter\def\csname PY@tok@nb\endcsname{\def\PY@tc##1{\textcolor[rgb]{0.00,0.50,0.00}{##1}}}
\expandafter\def\csname PY@tok@nf\endcsname{\def\PY@tc##1{\textcolor[rgb]{0.00,0.00,1.00}{##1}}}
\expandafter\def\csname PY@tok@nc\endcsname{\let\PY@bf=\textbf\def\PY@tc##1{\textcolor[rgb]{0.00,0.00,1.00}{##1}}}
\expandafter\def\csname PY@tok@nn\endcsname{\let\PY@bf=\textbf\def\PY@tc##1{\textcolor[rgb]{0.00,0.00,1.00}{##1}}}
\expandafter\def\csname PY@tok@ne\endcsname{\let\PY@bf=\textbf\def\PY@tc##1{\textcolor[rgb]{0.82,0.25,0.23}{##1}}}
\expandafter\def\csname PY@tok@nv\endcsname{\def\PY@tc##1{\textcolor[rgb]{0.10,0.09,0.49}{##1}}}
\expandafter\def\csname PY@tok@no\endcsname{\def\PY@tc##1{\textcolor[rgb]{0.53,0.00,0.00}{##1}}}
\expandafter\def\csname PY@tok@nl\endcsname{\def\PY@tc##1{\textcolor[rgb]{0.63,0.63,0.00}{##1}}}
\expandafter\def\csname PY@tok@ni\endcsname{\let\PY@bf=\textbf\def\PY@tc##1{\textcolor[rgb]{0.60,0.60,0.60}{##1}}}
\expandafter\def\csname PY@tok@na\endcsname{\def\PY@tc##1{\textcolor[rgb]{0.49,0.56,0.16}{##1}}}
\expandafter\def\csname PY@tok@nt\endcsname{\let\PY@bf=\textbf\def\PY@tc##1{\textcolor[rgb]{0.00,0.50,0.00}{##1}}}
\expandafter\def\csname PY@tok@nd\endcsname{\def\PY@tc##1{\textcolor[rgb]{0.67,0.13,1.00}{##1}}}
\expandafter\def\csname PY@tok@s\endcsname{\def\PY@tc##1{\textcolor[rgb]{0.73,0.13,0.13}{##1}}}
\expandafter\def\csname PY@tok@sd\endcsname{\let\PY@it=\textit\def\PY@tc##1{\textcolor[rgb]{0.73,0.13,0.13}{##1}}}
\expandafter\def\csname PY@tok@si\endcsname{\let\PY@bf=\textbf\def\PY@tc##1{\textcolor[rgb]{0.73,0.40,0.53}{##1}}}
\expandafter\def\csname PY@tok@se\endcsname{\let\PY@bf=\textbf\def\PY@tc##1{\textcolor[rgb]{0.73,0.40,0.13}{##1}}}
\expandafter\def\csname PY@tok@sr\endcsname{\def\PY@tc##1{\textcolor[rgb]{0.73,0.40,0.53}{##1}}}
\expandafter\def\csname PY@tok@ss\endcsname{\def\PY@tc##1{\textcolor[rgb]{0.10,0.09,0.49}{##1}}}
\expandafter\def\csname PY@tok@sx\endcsname{\def\PY@tc##1{\textcolor[rgb]{0.00,0.50,0.00}{##1}}}
\expandafter\def\csname PY@tok@m\endcsname{\def\PY@tc##1{\textcolor[rgb]{0.40,0.40,0.40}{##1}}}
\expandafter\def\csname PY@tok@gh\endcsname{\let\PY@bf=\textbf\def\PY@tc##1{\textcolor[rgb]{0.00,0.00,0.50}{##1}}}
\expandafter\def\csname PY@tok@gu\endcsname{\let\PY@bf=\textbf\def\PY@tc##1{\textcolor[rgb]{0.50,0.00,0.50}{##1}}}
\expandafter\def\csname PY@tok@gd\endcsname{\def\PY@tc##1{\textcolor[rgb]{0.63,0.00,0.00}{##1}}}
\expandafter\def\csname PY@tok@gi\endcsname{\def\PY@tc##1{\textcolor[rgb]{0.00,0.63,0.00}{##1}}}
\expandafter\def\csname PY@tok@gr\endcsname{\def\PY@tc##1{\textcolor[rgb]{1.00,0.00,0.00}{##1}}}
\expandafter\def\csname PY@tok@ge\endcsname{\let\PY@it=\textit}
\expandafter\def\csname PY@tok@gs\endcsname{\let\PY@bf=\textbf}
\expandafter\def\csname PY@tok@gp\endcsname{\let\PY@bf=\textbf\def\PY@tc##1{\textcolor[rgb]{0.00,0.00,0.50}{##1}}}
\expandafter\def\csname PY@tok@go\endcsname{\def\PY@tc##1{\textcolor[rgb]{0.53,0.53,0.53}{##1}}}
\expandafter\def\csname PY@tok@gt\endcsname{\def\PY@tc##1{\textcolor[rgb]{0.00,0.27,0.87}{##1}}}
\expandafter\def\csname PY@tok@err\endcsname{\def\PY@bc##1{\setlength{\fboxsep}{0pt}\fcolorbox[rgb]{1.00,0.00,0.00}{1,1,1}{\strut ##1}}}
\expandafter\def\csname PY@tok@kc\endcsname{\let\PY@bf=\textbf\def\PY@tc##1{\textcolor[rgb]{0.00,0.50,0.00}{##1}}}
\expandafter\def\csname PY@tok@kd\endcsname{\let\PY@bf=\textbf\def\PY@tc##1{\textcolor[rgb]{0.00,0.50,0.00}{##1}}}
\expandafter\def\csname PY@tok@kn\endcsname{\let\PY@bf=\textbf\def\PY@tc##1{\textcolor[rgb]{0.00,0.50,0.00}{##1}}}
\expandafter\def\csname PY@tok@kr\endcsname{\let\PY@bf=\textbf\def\PY@tc##1{\textcolor[rgb]{0.00,0.50,0.00}{##1}}}
\expandafter\def\csname PY@tok@bp\endcsname{\def\PY@tc##1{\textcolor[rgb]{0.00,0.50,0.00}{##1}}}
\expandafter\def\csname PY@tok@fm\endcsname{\def\PY@tc##1{\textcolor[rgb]{0.00,0.00,1.00}{##1}}}
\expandafter\def\csname PY@tok@vc\endcsname{\def\PY@tc##1{\textcolor[rgb]{0.10,0.09,0.49}{##1}}}
\expandafter\def\csname PY@tok@vg\endcsname{\def\PY@tc##1{\textcolor[rgb]{0.10,0.09,0.49}{##1}}}
\expandafter\def\csname PY@tok@vi\endcsname{\def\PY@tc##1{\textcolor[rgb]{0.10,0.09,0.49}{##1}}}
\expandafter\def\csname PY@tok@vm\endcsname{\def\PY@tc##1{\textcolor[rgb]{0.10,0.09,0.49}{##1}}}
\expandafter\def\csname PY@tok@sa\endcsname{\def\PY@tc##1{\textcolor[rgb]{0.73,0.13,0.13}{##1}}}
\expandafter\def\csname PY@tok@sb\endcsname{\def\PY@tc##1{\textcolor[rgb]{0.73,0.13,0.13}{##1}}}
\expandafter\def\csname PY@tok@sc\endcsname{\def\PY@tc##1{\textcolor[rgb]{0.73,0.13,0.13}{##1}}}
\expandafter\def\csname PY@tok@dl\endcsname{\def\PY@tc##1{\textcolor[rgb]{0.73,0.13,0.13}{##1}}}
\expandafter\def\csname PY@tok@s2\endcsname{\def\PY@tc##1{\textcolor[rgb]{0.73,0.13,0.13}{##1}}}
\expandafter\def\csname PY@tok@sh\endcsname{\def\PY@tc##1{\textcolor[rgb]{0.73,0.13,0.13}{##1}}}
\expandafter\def\csname PY@tok@s1\endcsname{\def\PY@tc##1{\textcolor[rgb]{0.73,0.13,0.13}{##1}}}
\expandafter\def\csname PY@tok@mb\endcsname{\def\PY@tc##1{\textcolor[rgb]{0.40,0.40,0.40}{##1}}}
\expandafter\def\csname PY@tok@mf\endcsname{\def\PY@tc##1{\textcolor[rgb]{0.40,0.40,0.40}{##1}}}
\expandafter\def\csname PY@tok@mh\endcsname{\def\PY@tc##1{\textcolor[rgb]{0.40,0.40,0.40}{##1}}}
\expandafter\def\csname PY@tok@mi\endcsname{\def\PY@tc##1{\textcolor[rgb]{0.40,0.40,0.40}{##1}}}
\expandafter\def\csname PY@tok@il\endcsname{\def\PY@tc##1{\textcolor[rgb]{0.40,0.40,0.40}{##1}}}
\expandafter\def\csname PY@tok@mo\endcsname{\def\PY@tc##1{\textcolor[rgb]{0.40,0.40,0.40}{##1}}}
\expandafter\def\csname PY@tok@ch\endcsname{\let\PY@it=\textit\def\PY@tc##1{\textcolor[rgb]{0.25,0.50,0.50}{##1}}}
\expandafter\def\csname PY@tok@cm\endcsname{\let\PY@it=\textit\def\PY@tc##1{\textcolor[rgb]{0.25,0.50,0.50}{##1}}}
\expandafter\def\csname PY@tok@cpf\endcsname{\let\PY@it=\textit\def\PY@tc##1{\textcolor[rgb]{0.25,0.50,0.50}{##1}}}
\expandafter\def\csname PY@tok@c1\endcsname{\let\PY@it=\textit\def\PY@tc##1{\textcolor[rgb]{0.25,0.50,0.50}{##1}}}
\expandafter\def\csname PY@tok@cs\endcsname{\let\PY@it=\textit\def\PY@tc##1{\textcolor[rgb]{0.25,0.50,0.50}{##1}}}

\def\PYZbs{\char`\\}
\def\PYZus{\char`\_}
\def\PYZob{\char`\{}
\def\PYZcb{\char`\}}
\def\PYZca{\char`\^}
\def\PYZam{\char`\&}
\def\PYZlt{\char`\<}
\def\PYZgt{\char`\>}
\def\PYZsh{\char`\#}
\def\PYZpc{\char`\%}
\def\PYZdl{\char`\$}
\def\PYZhy{\char`\-}
\def\PYZsq{\char`\'}
\def\PYZdq{\char`\"}
\def\PYZti{\char`\~}
% for compatibility with earlier versions
\def\PYZat{@}
\def\PYZlb{[}
\def\PYZrb{]}
\makeatother


    % Exact colors from NB
    \definecolor{incolor}{rgb}{0.0, 0.0, 0.5}
    \definecolor{outcolor}{rgb}{0.545, 0.0, 0.0}



    
    % Prevent overflowing lines due to hard-to-break entities
    \sloppy 
    % Setup hyperref package
    \hypersetup{
      breaklinks=true,  % so long urls are correctly broken across lines
      colorlinks=true,
      urlcolor=urlcolor,
      linkcolor=linkcolor,
      citecolor=citecolor,
      }
    % Slightly bigger margins than the latex defaults
    
    \geometry{verbose,tmargin=1in,bmargin=1in,lmargin=1in,rmargin=1in}
    
    

    \begin{document}
    
    
    \maketitle
    
    

    \begin{comment}
\title{Prétraitement - Importation et nettoyage de données}
    \author{
	Amina, KACIMI\\
	\and
	Valentin, DEHAINAULT\\
	\and
	Minh-Hoang, DANG\\	
}
\end{comment}
\begin{comment}
   \begin{Verbatim}[commandchars=\\\{\},fontsize=\scriptsize]
{\color{incolor}In [{\color{incolor}1}]:} \PY{c+c1}{\PYZhy{}\PYZhy{} connection: host=\PYZsq{}localhost\PYZsq{} dbname=\PYZsq{}Photographies\PYZsq{} user=\PYZsq{}postgres\PYZsq{}}
        \PY{k}{password}\PY{o}{=}\PY{l+s+s1}{\PYZsq{}}\PY{l+s+s1}{postgres}\PY{l+s+s1}{\PYZsq{}}
\end{Verbatim}
\end{comment}
    \hypertarget{importer-les-donuxe9es-brutes-dans-le-sgbd}{%
\section{Importer les donées brutes dans le
SGBD}\label{importer-les-donuxe9es-brutes-dans-le-sgbd}}

On a la possibilité d'importer les données à partir d'un fichier CSV
avec la requête \texttt{COPY\ ...\ FROM}

L'import des données va invoquer les règles définies sur la table cible.

\begin{verbatim}
COPY FROM will invoke any triggers and check constraints on the destination table. 
However, it will not invoke rules.
\end{verbatim}

Nous allons donc préparer deux tables vides, avec les attributs de types
adéquats, puis les règles pour les données entrantes. La première table,
nommée \texttt{CorrectionTemp} servira à la redirection des données vers
la vraie table \texttt{Correction}, via une multitude de triggers.
%\usepackage{verbatim}
\begin{comment}
   \begin{Verbatim}[commandchars=\\\{\},fontsize=\scriptsize]
{\color{incolor}In [{\color{incolor}2}]:} \PY{c+cm}{/*}\PY{c+cm}{DROP DATABASE IF EXISTS \PYZsq{}Photographies\PYZsq{};}
        \PY{c+cm}{CREATE DATABASE \PYZsq{}Photographies\PYZsq{}}
        \PY{c+cm}{    WITH}
        \PY{c+cm}{    OWNER = postgres}
        \PY{c+cm}{    ENCODING = \PYZsq{}UTF8\PYZsq{}}
        \PY{c+cm}{    LC\PYZus{}COLLATE = \PYZsq{}French\PYZus{}France.1252\PYZsq{}}
        \PY{c+cm}{    LC\PYZus{}CTYPE = \PYZsq{}French\PYZus{}France.1252\PYZsq{}}
        \PY{c+cm}{    TABLESPACE = pg\PYZus{}default}
        \PY{c+cm}{    CONNECTION LIMIT = \PYZhy{}1;}\PY{c+cm}{*/}
        
        \PY{k}{DROP} \PY{k}{TABLE} \PY{k}{IF} \PY{k}{EXISTS} \PY{n}{CorrectionTemp}\PY{p}{;}
        \PY{k}{CREATE} \PY{k}{TABLE} \PY{n}{CorrectionTemp}\PY{p}{(}
            \PY{n}{ReferenceCindoc} \PY{n+nb}{varchar} \PY{k}{NULL} \PY{p}{,}
            \PY{n}{Serie} \PY{n+nb}{varchar} \PY{k}{NULL} \PY{p}{,}
            \PY{n}{Article} \PY{n+nb}{varchar} \PY{k}{NULL}\PY{p}{,}
            \PY{n}{Discriminant} \PY{n+nb}{varchar} \PY{k}{NULL}\PY{p}{,}
            \PY{n}{Ville} \PY{n+nb}{varchar} \PY{k}{NULL}\PY{p}{,}
            \PY{n}{Sujet} \PY{n+nb}{varchar} \PY{k}{NULL}\PY{p}{,}
            \PY{n}{DescDet} \PY{n+nb}{varchar} \PY{k}{NULL}\PY{p}{,}
            \PY{n+nb}{Date} \PY{n+nb}{varchar} \PY{k}{NULL}\PY{p}{,}
            \PY{n}{Notebp} \PY{n+nb}{varchar} \PY{k}{NULL}\PY{p}{,}
            \PY{n}{FicNum} \PY{n+nb}{varchar} \PY{k}{NULL}\PY{p}{,}
            \PY{n}{Idx\PYZus{}Ico} \PY{n+nb}{varchar} \PY{k}{NULL}\PY{p}{,}
            \PY{n}{NbrCli} \PY{n+nb}{varchar} \PY{k}{NULL}\PY{p}{,}
            \PY{n}{TailleCli} \PY{n+nb}{varchar} \PY{k}{NULL}\PY{p}{,}
            \PY{n}{Taille} \PY{n+nb}{varchar} \PY{k}{NULL}\PY{p}{,}
            \PY{n}{N\PYZus{}V} \PY{n+nb}{varchar} \PY{k}{NULL}\PY{p}{,}
            \PY{n}{C\PYZus{}G} \PY{n+nb}{varchar} \PY{k}{NULL}\PY{p}{,}
            \PY{n}{Remarques} \PY{n+nb}{varchar} \PY{k}{NULL}
        \PY{p}{)}\PY{p}{;}
        
        
        
        \PY{k}{DROP} \PY{k}{TABLE} \PY{k}{IF} \PY{k}{EXISTS} \PY{n}{Correction}\PY{p}{;}
        \PY{k}{CREATE} \PY{k}{TABLE} \PY{n}{Correction}\PY{p}{(}
            \PY{n}{ReferenceCindoc} \PY{n+nb}{varchar}\PY{p}{(}\PY{l+m+mf}{6}\PY{p}{)}\PY{p}{,} \PY{c+c1}{\PYZhy{}\PYZhy{}SEP |}
            \PY{n}{Serie} \PY{n+nb}{varchar} \PY{k}{NULL} \PY{p}{,}
            \PY{n}{Article} \PY{n+nb}{int} \PY{k}{NULL}\PY{p}{,} \PY{c+c1}{\PYZhy{}\PYZhy{}AUTO INCREMENT, PRIMARY KEY}
            \PY{n}{Discriminant} \PY{n+nb}{varchar} \PY{k}{NULL}\PY{p}{,}
            \PY{n}{Ville} \PY{n+nb}{varchar} \PY{k}{NULL}\PY{p}{,}
            \PY{n}{Sujet} \PY{n+nb}{varchar} \PY{k}{NULL}\PY{p}{,} \PY{c+c1}{\PYZhy{}\PYZhy{} SEP ,}
            \PY{n}{DescDet} \PY{n+nb}{varchar} \PY{k}{NULL}\PY{p}{,}
            \PY{n+nb}{Date} \PY{n+nb}{varchar} \PY{k}{NULL}\PY{p}{,} \PY{c+c1}{\PYZhy{}\PYZhy{} remove first chunk, parse date from french}
            \PY{n}{NoteBP} \PY{n+nb}{varchar} \PY{k}{NULL}\PY{p}{,} \PY{c+c1}{\PYZhy{}\PYZhy{} SEP /}
            \PY{n}{FicNum} \PY{n+nb}{varchar} \PY{k}{NULL}\PY{p}{,} \PY{c+c1}{\PYZhy{}\PYZhy{} SEP /}
            \PY{n}{Idx\PYZus{}Ico} \PY{n+nb}{varchar} \PY{k}{NULL}\PY{p}{,} \PY{c+c1}{\PYZhy{}\PYZhy{} SEP /, regex CCCCDDD\PYZus{}[SERIE]\PYZus{}DDDDDD\PYZus{}D}
            \PY{n}{NbrCli} \PY{n+nb}{varchar} \PY{k}{NULL}\PY{p}{,} \PY{c+c1}{\PYZhy{}\PYZhy{} SEP /, |}
            \PY{n}{TailleCli} \PY{n+nb}{varchar} \PY{k}{NULL}\PY{p}{,} \PY{c+c1}{\PYZhy{}\PYZhy{}SEP, par NbrCli}
            \PY{n}{Taille} \PY{n+nb}{varchar} \PY{k}{NULL}\PY{p}{,} \PY{c+c1}{\PYZhy{}\PYZhy{} FNable}
            \PY{n}{N\PYZus{}V} \PY{n+nb}{varchar} \PY{k}{NULL}\PY{p}{,} \PY{c+c1}{\PYZhy{}\PYZhy{} regex NEG|INV}
            \PY{n}{C\PYZus{}G} \PY{n+nb}{varchar} \PY{k}{NULL}\PY{p}{,} \PY{c+c1}{\PYZhy{}\PYZhy{} regex \PYZsq{}GSC\PYZsq{}|\PYZsq{}CLR\PYZsq{}}
            \PY{n}{Remarques} \PY{n+nb}{varchar} \PY{k}{NULL} \PY{c+c1}{\PYZhy{}\PYZhy{} S}
        \PY{p}{)}\PY{p}{;}
\end{Verbatim}
\end{comment}
    \hypertarget{nettoyage-de-donnuxe9es}{%
\section{Nettoyage de données}\label{nettoyage-de-donnuxe9es}}

Comme nous l'avons discuté précédemment, nous allons écrire un
\texttt{TRIGGER} pour chaque `problème' ci-dessous. Afin d'obtenir le
résultat le plus satisfaisant, nous avons besoins d'une librairie de
fonctions qui permettent de traiter les données entrantes sous forme de
\texttt{string}.
% Please add the following required packages to your document preamble:
% \usepackage{graphicx}
% \usepackage{float}
% \restylefloat{table}


\begin{table}[H]
\resizebox{\textwidth}{!}{%
\begin{tabular}{|l|l|}
\hline
Fonction                                                                                             & But                                                                                                                                                                                                                \\ \hline
split\_string(str,delim1, delim2)                                                                    & Séparer un string délimité par deux délimiteurs en un ensemble de strings                                                                                                                                          \\ \hline
\begin{tabular}[c]{@{}l@{}}traitement\_date(str)format d’entrée : \\ année / mois année\end{tabular} & \begin{tabular}[c]{@{}l@{}}Retourner un string contenant uniquement \\ le mois si précisé et l’année de prise d’une photo\end{tabular}                                                                             \\ \hline
traitement\_n\_v(str)                                                                                & \begin{tabular}[c]{@{}l@{}}Harmoniser la colonne négative ou inversible de manière \\ à ce qu’elle contienne NEG pour négatif et INV pour inversible\end{tabular}                                                  \\ \hline
array\_expand(array, maxlength, fill)                                                                & \begin{tabular}[c]{@{}l@{}}Étendre le array jusqu’à certaine taille en ajoutant les éléments. \\ Cette fonction est particulièrement utile pour combler les vides \\ après la séparation des données.\end{tabular} \\ \hline
\end{tabular}%
}
\end{table}%\usepackage{verbatim}
\begin{comment}
   \begin{Verbatim}[commandchars=\\\{\},fontsize=\scriptsize]
{\color{incolor}In [{\color{incolor}3}]:} \PY{c+cm}{/*}\PY{c+cm}{*}
        \PY{c+cm}{ }\PY{c+cm}{*}\PY{c+cm}{ COMPOSITE TYPE: insert\PYZus{}array}
        \PY{c+cm}{ }\PY{c+cm}{*}\PY{c+cm}{ C\PYZsq{}est un type qui contient un array de varchar}
        \PY{c+cm}{ }\PY{c+cm}{*}\PY{c+cm}{ Le champ acceuillit les données préparées pour l\PYZsq{}insertion}
        \PY{c+cm}{ }\PY{c+cm}{*/}
        \PY{k}{DROP} \PY{k}{TYPE} \PY{k}{IF} \PY{k}{EXISTS} \PY{n}{insert\PYZus{}array} \PY{k}{CASCADE}\PY{p}{;}
        \PY{k}{CREATE} \PY{k}{TYPE} \PY{n}{insert\PYZus{}array} \PY{k}{AS} \PY{p}{(}
            \PY{n}{field} \PY{n+nb}{varchar}\PY{p}{[}\PY{p}{]}
        \PY{p}{)}\PY{p}{;}
        
        
        \PY{c+cm}{/*}\PY{c+cm}{*}
        \PY{c+cm}{ }\PY{c+cm}{*}\PY{c+cm}{ FUNCTION: split\PYZus{}string(str, delim1, delim2)}
        \PY{c+cm}{ }\PY{c+cm}{*}\PY{c+cm}{ Séparer un string par 2 delimiteurs en un ensemble de strings}
        \PY{c+cm}{ }\PY{c+cm}{*/}
        \PY{k}{DROP} \PY{k}{FUNCTION} \PY{k}{IF} \PY{k}{EXISTS} \PY{n}{split\PYZus{}string}\PY{p}{;}
        \PY{k}{CREATE} \PY{k}{OR} \PY{k}{REPLACE} \PY{k}{FUNCTION} \PY{n}{public}\PY{l+m+mf}{.}\PY{n}{split\PYZus{}string}\PY{p}{(}\PY{n}{\PYZus{}str} \PY{n+nb}{varchar}\PY{p}{,} \PY{n}{\PYZus{}delim1} \PY{n+nb}{varchar}\PY{p}{,} \PY{n}{\PYZus{}delim2}
        \PY{n+nb}{varchar}\PY{p}{)}
        \PY{k}{RETURNS} \PY{n+nb}{varchar}\PY{p}{[}\PY{p}{]} \PY{k}{AS} \PY{l+s}{\PYZdl{}}\PY{l+s}{\PYZdl{}}
            \PY{k}{DECLARE}
                \PY{n}{result} \PY{n+nb}{varchar}\PY{p}{[}\PY{p}{]} \PY{p}{;}
            \PY{k}{BEGIN}
                \PY{n}{result} \PY{o}{:=} \PY{k}{ARRAY}\PY{p}{(}
                    \PY{k}{SELECT} \PY{n}{unnest}\PY{p}{(}\PY{n}{string\PYZus{}to\PYZus{}array}\PY{p}{(}\PY{n}{a}\PY{p}{,} \PY{n}{\PYZus{}delim2}\PY{p}{)}\PY{p}{)}
                    \PY{k}{FROM}   \PY{n}{unnest}\PY{p}{(}\PY{n}{string\PYZus{}to\PYZus{}array}\PY{p}{(}\PY{n}{\PYZus{}str}\PY{p}{,} \PY{n}{\PYZus{}delim1}\PY{p}{)}\PY{p}{)} \PY{n}{a}
                \PY{p}{)}\PY{p}{;}
                \PY{c+c1}{\PYZhy{}\PYZhy{}RAISE NOTICE \PYZsq{}split\PYZus{}string(): \PYZpc{}\PYZsq{}, result;}
                \PY{k}{RETURN} \PY{n}{result}\PY{p}{;}
            \PY{k}{END}\PY{p}{;}
        \PY{l+s}{\PYZdl{}}\PY{l+s}{\PYZdl{}} \PY{k}{LANGUAGE} \PY{n}{plpgsql}\PY{p}{;}
        
        \PY{c+cm}{/*}\PY{c+cm}{*}
        \PY{c+cm}{ }\PY{c+cm}{*}\PY{c+cm}{ FUNCTION: traitement\PYZus{}date(str)}
        \PY{c+cm}{ }\PY{c+cm}{*}\PY{c+cm}{  input format \PYZsq{}chunk: [month year | year ]}
        \PY{c+cm}{ }\PY{c+cm}{*/}
        \PY{k}{DROP} \PY{k}{FUNCTION} \PY{k}{IF} \PY{k}{EXISTS} \PY{n}{traitement\PYZus{}date}\PY{p}{;}
        \PY{k}{CREATE} \PY{k}{OR} \PY{k}{REPLACE} \PY{k}{FUNCTION} \PY{n}{traitement\PYZus{}date}\PY{p}{(}\PY{n}{\PYZus{}str} \PY{n+nb}{varchar}\PY{p}{)}
        \PY{k}{RETURNS} \PY{n+nb}{varchar} \PY{k}{AS} \PY{l+s}{\PYZdl{}}\PY{l+s}{\PYZdl{}}
            \PY{k}{BEGIN}
                \PY{k}{RETURN} \PY{n}{split\PYZus{}part}\PY{p}{(}\PY{n}{\PYZus{}str}\PY{p}{,} \PY{l+s+s1}{\PYZsq{}}\PY{l+s+s1}{:}\PY{l+s+s1}{\PYZsq{}}\PY{p}{,} \PY{l+m+mf}{2}\PY{p}{)}\PY{p}{;}
            \PY{k}{END}\PY{p}{;}
        \PY{l+s}{\PYZdl{}}\PY{l+s}{\PYZdl{}} \PY{k}{LANGUAGE} \PY{n}{plpgsql}\PY{p}{;}
        
        \PY{c+cm}{/*}\PY{c+cm}{*}
        \PY{c+cm}{ }\PY{c+cm}{*}\PY{c+cm}{ FUNCTION: traitement\PYZus{}n\PYZus{}v(str)}
        \PY{c+cm}{ }\PY{c+cm}{*}\PY{c+cm}{  négatif = NEG, inversible = INV, details += remarques}
        \PY{c+cm}{ }\PY{c+cm}{*/}
        \PY{k}{DROP} \PY{k}{FUNCTION} \PY{k}{IF} \PY{k}{EXISTS} \PY{n}{traitement\PYZus{}n\PYZus{}v}\PY{p}{;}
        \PY{k}{CREATE} \PY{k}{OR} \PY{k}{REPLACE} \PY{k}{FUNCTION} \PY{n}{traitement\PYZus{}n\PYZus{}v}\PY{p}{(} \PY{n}{strSet} \PY{n+nb}{varchar}\PY{p}{[}\PY{p}{]}\PY{p}{)}
        \PY{k}{RETURNS} \PY{n+nb}{varchar}\PY{p}{(}\PY{l+m+mf}{3}\PY{p}{)}\PY{p}{[}\PY{p}{]} \PY{k}{AS} \PY{l+s}{\PYZdl{}}\PY{l+s}{\PYZdl{}}
            \PY{k}{DECLARE}
                \PY{n}{i} \PY{n+nb}{int}\PY{p}{;}
            \PY{k}{BEGIN}
                \PY{k}{IF} \PY{n}{strSet} \PY{k}{is} \PY{k}{NULL} \PY{k}{THEN}
                    \PY{k}{RETURN} \PY{k}{NULL}\PY{p}{;}
                \PY{k}{END} \PY{k}{IF}\PY{p}{;}
        
                \PY{k}{FOR} \PY{n}{i} \PY{k}{IN} \PY{l+m+mf}{1} \PY{l+m+mf}{.}\PY{l+m+mf}{.} \PY{n}{array\PYZus{}length}\PY{p}{(}\PY{n}{strSet}\PY{p}{,} \PY{l+m+mf}{1}\PY{p}{)} \PY{k}{LOOP}
                    \PY{k}{IF} \PY{p}{(}\PY{n}{strSet}\PY{p}{[}\PY{n}{i}\PY{p}{]} \PY{o}{\PYZti{}*} \PY{l+s+s1}{\PYZsq{}}\PY{l+s+s1}{.*négatif.*}\PY{l+s+s1}{\PYZsq{}}\PY{p}{)} \PY{k}{THEN}
                        \PY{n}{strSet}\PY{p}{[}\PY{n}{i}\PY{p}{]} \PY{o}{:=} \PY{l+s+s1}{\PYZsq{}}\PY{l+s+s1}{NEG}\PY{l+s+s1}{\PYZsq{}}\PY{p}{;}
                    \PY{k}{ELSE}
                        \PY{n}{strSet}\PY{p}{[}\PY{n}{i}\PY{p}{]} \PY{o}{:=} \PY{l+s+s1}{\PYZsq{}}\PY{l+s+s1}{INV}\PY{l+s+s1}{\PYZsq{}}\PY{p}{;}
                    \PY{k}{END} \PY{k}{IF}\PY{p}{;}
                \PY{k}{END} \PY{k}{LOOP}\PY{p}{;}
                \PY{k}{RETURN} \PY{n}{strSet}\PY{p}{;}
            \PY{k}{END}\PY{p}{;}
        \PY{l+s}{\PYZdl{}}\PY{l+s}{\PYZdl{}} \PY{k}{LANGUAGE} \PY{n}{plpgsql}\PY{p}{;}
        
        \PY{c+cm}{/*}\PY{c+cm}{*}
        \PY{c+cm}{ }\PY{c+cm}{*}\PY{c+cm}{ FUNCTION: array\PYZus{}expand()}
        \PY{c+cm}{ }\PY{c+cm}{*}\PY{c+cm}{  négatif = NEG, inversible = INV, details += remarques}
        \PY{c+cm}{ }\PY{c+cm}{*/}
        \PY{k}{DROP} \PY{k}{FUNCTION} \PY{k}{IF} \PY{k}{EXISTS} \PY{n}{array\PYZus{}expand}\PY{p}{;}
        \PY{k}{CREATE} \PY{k}{OR} \PY{k}{REPLACE} \PY{k}{FUNCTION} \PY{n}{array\PYZus{}expand}\PY{p}{(} \PY{n}{arr} \PY{n+nb}{anyarray}\PY{p}{,} \PY{n}{maxlength} \PY{n+nb}{int}\PY{p}{,} \PY{n}{fill} \PY{n+nb}{anyelement} \PY{o}{=}
        \PY{k}{null} \PY{p}{)}
        \PY{k}{RETURNS} \PY{n+nb}{anyarray} \PY{k}{AS} \PY{l+s}{\PYZdl{}}\PY{l+s}{\PYZdl{}}
            \PY{k}{DECLARE}
                \PY{n}{i} \PY{n+nb}{int}\PY{p}{;}
                \PY{n}{length} \PY{n+nb}{int} \PY{o}{:=} \PY{k}{coalesce}\PY{p}{(}\PY{n}{array\PYZus{}length}\PY{p}{(}\PY{n}{arr}\PY{p}{,} \PY{l+m+mf}{1}\PY{p}{)}\PY{p}{,} \PY{l+m+mf}{0}\PY{p}{)}\PY{p}{;}
            \PY{k}{BEGIN}
                \PY{c+c1}{\PYZhy{}\PYZhy{}RAISE NOTICE \PYZsq{}MaxLength: \PYZpc{}, Length: \PYZpc{}\PYZsq{}, maxlength, length;}
                \PY{k}{IF} \PY{p}{(}\PY{n}{maxlength} \PY{o}{\PYZgt{}} \PY{n}{length}\PY{p}{)} \PY{k}{THEN}
                    \PY{k}{FOR} \PY{n}{i} \PY{k}{IN} \PY{l+m+mf}{1} \PY{l+m+mf}{.}\PY{l+m+mf}{.} \PY{p}{(}\PY{n}{maxlength} \PY{o}{\PYZhy{}} \PY{n}{length}\PY{p}{)} \PY{k}{LOOP}
                        \PY{n}{arr} \PY{o}{:=} \PY{n}{array\PYZus{}append}\PY{p}{(}\PY{n}{arr}\PY{p}{,} \PY{n}{fill}\PY{p}{)}\PY{p}{;}
                    \PY{k}{END} \PY{k}{LOOP}\PY{p}{;}
                \PY{k}{END} \PY{k}{IF}\PY{p}{;}
                \PY{k}{RETURN} \PY{n}{arr}\PY{p}{;}
            \PY{k}{END}\PY{p}{;}
        \PY{l+s}{\PYZdl{}}\PY{l+s}{\PYZdl{}} \PY{k}{LANGUAGE} \PY{n}{plpgsql}\PY{p}{;}
        
        \PY{c+cm}{/*}\PY{c+cm}{*}
        \PY{c+cm}{ }\PY{c+cm}{*}\PY{c+cm}{ FUNCTION: traitement\PYZus{}n\PYZus{}v(str)}
        \PY{c+cm}{ }\PY{c+cm}{*}\PY{c+cm}{  négatif = NEG, inversible = INV, details += remarques}
        \PY{c+cm}{ }\PY{c+cm}{*/}
        \PY{k}{DROP} \PY{k}{FUNCTION} \PY{k}{IF} \PY{k}{EXISTS} \PY{n}{getMaxLength}\PY{p}{;}
        \PY{k}{CREATE} \PY{k}{OR} \PY{k}{REPLACE} \PY{k}{FUNCTION} \PY{n}{getMaxLength}\PY{p}{(}\PY{k}{VARIADIC} \PY{n}{arr} \PY{n}{insert\PYZus{}array}\PY{p}{[}\PY{p}{]} \PY{p}{)}
        \PY{k}{RETURNS} \PY{n+nb}{int} \PY{k}{AS} \PY{l+s}{\PYZdl{}}\PY{l+s}{\PYZdl{}}
            \PY{k}{DECLARE}
                \PY{n}{result} \PY{n+nb}{int}\PY{p}{;}
            \PY{k}{BEGIN}
                \PY{k}{SELECT} \PY{k}{coalesce}\PY{p}{(}\PY{n}{max}\PY{p}{(}\PY{n}{array\PYZus{}length}\PY{p}{(}\PY{n+nv}{\PYZdl{}1}\PY{p}{[}\PY{n}{i}\PY{p}{]}\PY{l+m+mf}{.}\PY{n}{field}\PY{p}{,} \PY{l+m+mf}{1}\PY{p}{)}\PY{p}{)}\PY{p}{,} \PY{l+m+mf}{0}\PY{p}{)}
                \PY{k}{FROM} \PY{n}{generate\PYZus{}subscripts}\PY{p}{(}\PY{n+nv}{\PYZdl{}1}\PY{p}{,} \PY{l+m+mf}{1}\PY{p}{)} \PY{n}{g}\PY{p}{(}\PY{n}{i}\PY{p}{)}
                \PY{k}{INTO} \PY{n}{result}\PY{p}{;}
                \PY{c+c1}{\PYZhy{}\PYZhy{}RAISE NOTICE \PYZsq{}maxLength: \PYZpc{}\PYZsq{}, result;}
                \PY{k}{RETURN} \PY{n}{result}\PY{p}{;}
            \PY{k}{END}\PY{p}{;}
        \PY{l+s}{\PYZdl{}}\PY{l+s}{\PYZdl{}} \PY{k}{LANGUAGE} \PY{n}{plpgsql}\PY{p}{;}
\end{Verbatim}

    \begin{Verbatim}[commandchars=\\\{\},fontsize=\footnotesize]
NOTICE:  drop cascades to function getmaxlength(insert\_array[])
NOTICE:  function getmaxlength() does not exist, skipping

    \end{Verbatim}
\end{comment}
    \hypertarget{uxe9liminer-les-incohuxe9rences-dans-les-donnuxe9es}{%
\subsection{Éliminer les incohérences dans les
données}\label{uxe9liminer-les-incohuxe9rences-dans-les-donnuxe9es}}

Nous avons remarqué une certaine incohérence dans les données:
plusieures syntaxes pour exprimer une information (négatif/verre
négatif), plusieurs types de tâches ou d'usure référencées, \ldots{} Ce
traitement est chargé dans le trigger qui fait la séparation des lignes.

Pour supprimer les doublons, on procède de la manière suivante: On
vérifie si deux lignes sont identiques, plus précisément si toutes les
valeurs contenues dans une ligne sont identiques à celles d'une autre si
c'est le cas on supprime.

    \hypertarget{suxe9parer-les-lignes-combinuxe9es}{%
\subsection{Séparer les lignes
combinées}\label{suxe9parer-les-lignes-combinuxe9es}}

Pour séparer les lignes combinées, nous avons crée un trigger qui
détecte la présence des séparateurs (virgules, pipe, slash) selon chaque
attribut du fichier CSV (donc la présence de multiples informations dans
une même case). Avec les délimiteurs, on sépare les informations
combinées dans un array correspondant à l'attribut concerné. Finalement,
on passe les attributs dans la clause \texttt{VALUES} de
\texttt{INSERT}.
%\usepackage{verbatim}
\begin{comment}
   \begin{Verbatim}[commandchars=\\\{\},fontsize=\scriptsize]
{\color{incolor}In [{\color{incolor}4}]:} \PY{c+cm}{/*}\PY{c+cm}{*}
        \PY{c+cm}{ }\PY{c+cm}{*}\PY{c+cm}{ TRIGGER FUNCTION: separer\PYZus{}ligne\PYZus{}combinees}
        \PY{c+cm}{ }\PY{c+cm}{*}\PY{c+cm}{ Détecter les lignes combinées, séparer les bouts et}
        \PY{c+cm}{ }\PY{c+cm}{*}\PY{c+cm}{ faire les insertions (pas nécessairement propre)}
        \PY{c+cm}{ }\PY{c+cm}{*/}
        \PY{k}{CREATE} \PY{k}{OR} \PY{k}{REPLACE} \PY{k}{FUNCTION} \PY{n}{separer\PYZus{}ligne\PYZus{}combinees}\PY{p}{(}\PY{p}{)}
        \PY{k}{RETURNS} \PY{k}{TRIGGER} \PY{k}{AS} \PY{l+s}{\PYZdl{}}\PY{l+s}{\PYZdl{}}
            \PY{k}{DECLARE}
                \PY{n}{ReferenceCindocVals} \PY{n}{insert\PYZus{}array}\PY{p}{;}
                \PY{n}{DiscriminantVals} \PY{n}{insert\PYZus{}array}\PY{p}{;}
                \PY{n}{VilleVals} \PY{n}{insert\PYZus{}array}\PY{p}{;}
                \PY{n}{SujetVals} \PY{n}{insert\PYZus{}array}\PY{p}{;}
                \PY{n}{NoteBPVals} \PY{n}{insert\PYZus{}array}\PY{p}{;}
                \PY{n}{FicNumVals} \PY{n}{insert\PYZus{}array}\PY{p}{;}
                \PY{n}{Idx\PYZus{}IcoVals} \PY{n}{insert\PYZus{}array}\PY{p}{;}
                \PY{n}{NbrCliVals} \PY{n}{insert\PYZus{}array}\PY{p}{;}
                \PY{n}{TailleCliVals} \PY{n}{insert\PYZus{}array}\PY{p}{;}
                \PY{n}{N\PYZus{}VVals} \PY{n}{insert\PYZus{}array}\PY{p}{;}
                \PY{n}{C\PYZus{}GVals} \PY{n}{insert\PYZus{}array}\PY{p}{;}
        
                \PY{n}{maxLength} \PY{n+nb}{int}\PY{p}{;}
        
                \PY{n}{insertVals} \PY{n}{insert\PYZus{}array}\PY{p}{[}\PY{p}{]}\PY{p}{;}
        
            \PY{k}{BEGIN}
                \PY{c+c1}{\PYZhy{}\PYZhy{}\PYZhy{}\PYZhy{}\PYZhy{}\PYZhy{}\PYZhy{}\PYZhy{}\PYZhy{}\PYZhy{}\PYZhy{}\PYZhy{}\PYZhy{}\PYZhy{}\PYZhy{}\PYZhy{}\PYZhy{}\PYZhy{}\PYZhy{}\PYZhy{}\PYZhy{}\PYZhy{}\PYZhy{}\PYZhy{}\PYZhy{}\PYZhy{}\PYZhy{}\PYZhy{}\PYZhy{}\PYZhy{}\PYZhy{}\PYZhy{}\PYZhy{}\PYZhy{}\PYZhy{}\PYZhy{}\PYZhy{}\PYZhy{}\PYZhy{}\PYZhy{}\PYZhy{}\PYZhy{}\PYZhy{}\PYZhy{}\PYZhy{}\PYZhy{}\PYZhy{}\PYZhy{}\PYZhy{}\PYZhy{}\PYZhy{}\PYZhy{}\PYZhy{}\PYZhy{}\PYZhy{}\PYZhy{}\PYZhy{}\PYZhy{}\PYZhy{}\PYZhy{}\PYZhy{}\PYZhy{}\PYZhy{}\PYZhy{}\PYZhy{}\PYZhy{}\PYZhy{}\PYZhy{}\PYZhy{}\PYZhy{}\PYZhy{}\PYZhy{}\PYZhy{}\PYZhy{}\PYZhy{}\PYZhy{}\PYZhy{}}
                \PY{c+c1}{\PYZhy{}\PYZhy{} Traitement des données}
                \PY{c+c1}{\PYZhy{}\PYZhy{}\PYZhy{}\PYZhy{}\PYZhy{}\PYZhy{}\PYZhy{}\PYZhy{}\PYZhy{}\PYZhy{}\PYZhy{}\PYZhy{}\PYZhy{}\PYZhy{}\PYZhy{}\PYZhy{}\PYZhy{}\PYZhy{}\PYZhy{}\PYZhy{}\PYZhy{}\PYZhy{}\PYZhy{}\PYZhy{}\PYZhy{}\PYZhy{}\PYZhy{}\PYZhy{}\PYZhy{}\PYZhy{}\PYZhy{}\PYZhy{}\PYZhy{}\PYZhy{}\PYZhy{}\PYZhy{}\PYZhy{}\PYZhy{}\PYZhy{}\PYZhy{}\PYZhy{}\PYZhy{}\PYZhy{}\PYZhy{}\PYZhy{}\PYZhy{}\PYZhy{}\PYZhy{}\PYZhy{}\PYZhy{}\PYZhy{}\PYZhy{}\PYZhy{}\PYZhy{}\PYZhy{}\PYZhy{}\PYZhy{}\PYZhy{}\PYZhy{}\PYZhy{}\PYZhy{}\PYZhy{}\PYZhy{}\PYZhy{}\PYZhy{}\PYZhy{}\PYZhy{}\PYZhy{}\PYZhy{}\PYZhy{}\PYZhy{}\PYZhy{}\PYZhy{}\PYZhy{}\PYZhy{}\PYZhy{}\PYZhy{}}
                \PY{n}{ReferenceCindocVals}\PY{l+m+mf}{.}\PY{n}{field} \PY{o}{:=} \PY{n}{string\PYZus{}to\PYZus{}array}\PY{p}{(}\PY{k}{replace}\PY{p}{(}\PY{n}{NEW}\PY{l+m+mf}{.}\PY{n}{ReferenceCindoc}\PY{p}{,} \PY{l+s+s1}{\PYZsq{}}\PY{l+s+s1}{ }\PY{l+s+s1}{\PYZsq{}}\PY{p}{,}
        \PY{l+s+s1}{\PYZsq{}}\PY{l+s+s1}{\PYZsq{}}\PY{p}{)}\PY{p}{,} \PY{l+s+s1}{\PYZsq{}}\PY{l+s+s1}{|}\PY{l+s+s1}{\PYZsq{}}\PY{p}{)}\PY{p}{;}
                \PY{n}{DiscriminantVals}\PY{l+m+mf}{.}\PY{n}{field} \PY{o}{:=} \PY{n}{string\PYZus{}to\PYZus{}array}\PY{p}{(}\PY{n}{NEW}\PY{l+m+mf}{.}\PY{n}{Discriminant}\PY{p}{,} \PY{l+s+s1}{\PYZsq{}}\PY{l+s+s1}{|}\PY{l+s+s1}{\PYZsq{}}\PY{p}{)}\PY{p}{;}
                \PY{n}{VilleVals}\PY{l+m+mf}{.}\PY{n}{field} \PY{o}{:=} \PY{n}{string\PYZus{}to\PYZus{}array}\PY{p}{(}\PY{n}{NEW}\PY{l+m+mf}{.}\PY{n}{Ville}\PY{p}{,} \PY{l+s+s1}{\PYZsq{}}\PY{l+s+s1}{,}\PY{l+s+s1}{\PYZsq{}}\PY{p}{)}\PY{p}{;}
                \PY{n}{SujetVals}\PY{l+m+mf}{.}\PY{n}{field} \PY{o}{:=} \PY{n}{string\PYZus{}to\PYZus{}array}\PY{p}{(}\PY{n}{NEW}\PY{l+m+mf}{.}\PY{n}{sujet}\PY{p}{,} \PY{l+s+s1}{\PYZsq{}}\PY{l+s+s1}{,}\PY{l+s+s1}{\PYZsq{}}\PY{p}{)}\PY{p}{;}
                \PY{n}{NoteBPVals}\PY{l+m+mf}{.}\PY{n}{field} \PY{o}{:=} \PY{n}{string\PYZus{}to\PYZus{}array}\PY{p}{(}\PY{n}{NEW}\PY{l+m+mf}{.}\PY{n}{NoteBP}\PY{p}{,} \PY{l+s+s1}{\PYZsq{}}\PY{l+s+s1}{/}\PY{l+s+s1}{\PYZsq{}}\PY{p}{)}\PY{p}{;}
                \PY{n}{FicNumVals}\PY{l+m+mf}{.}\PY{n}{field} \PY{o}{:=} \PY{n}{string\PYZus{}to\PYZus{}array}\PY{p}{(}\PY{n}{NEW}\PY{l+m+mf}{.}\PY{n}{FicNum}\PY{p}{,} \PY{l+s+s1}{\PYZsq{}}\PY{l+s+s1}{/}\PY{l+s+s1}{\PYZsq{}}\PY{p}{)}\PY{p}{;}
                \PY{n}{Idx\PYZus{}IcoVals}\PY{l+m+mf}{.}\PY{n}{field} \PY{o}{:=} \PY{n}{string\PYZus{}to\PYZus{}array}\PY{p}{(}\PY{n}{NEW}\PY{l+m+mf}{.}\PY{n}{Idx\PYZus{}ico}\PY{p}{,} \PY{l+s+s1}{\PYZsq{}}\PY{l+s+s1}{/}\PY{l+s+s1}{\PYZsq{}}\PY{p}{)}\PY{p}{;}
                \PY{n}{NbrCliVals}\PY{l+m+mf}{.}\PY{n}{field} \PY{o}{:=} \PY{n}{split\PYZus{}string}\PY{p}{(}\PY{n}{NEW}\PY{l+m+mf}{.}\PY{n}{NbrCli}\PY{p}{,} \PY{l+s+s1}{\PYZsq{}}\PY{l+s+s1}{|}\PY{l+s+s1}{\PYZsq{}}\PY{p}{,} \PY{l+s+s1}{\PYZsq{}}\PY{l+s+s1}{/}\PY{l+s+s1}{\PYZsq{}}\PY{p}{)}\PY{p}{;}
                \PY{n}{TailleCliVals}\PY{l+m+mf}{.}\PY{n}{field} \PY{o}{:=} \PY{n}{string\PYZus{}to\PYZus{}array}\PY{p}{(}\PY{n}{NEW}\PY{l+m+mf}{.}\PY{n}{TailleCli}\PY{p}{,} \PY{l+s+s1}{\PYZsq{}}\PY{l+s+s1}{,}\PY{l+s+s1}{\PYZsq{}}\PY{p}{)}\PY{p}{;}
                \PY{n}{N\PYZus{}VVals}\PY{l+m+mf}{.}\PY{n}{field} \PY{o}{:=} \PY{n}{traitement\PYZus{}n\PYZus{}v}\PY{p}{(}\PY{n}{string\PYZus{}to\PYZus{}array}\PY{p}{(}\PY{n}{NEW}\PY{l+m+mf}{.}\PY{n}{n\PYZus{}v}\PY{p}{,} \PY{l+s+s1}{\PYZsq{}}\PY{l+s+s1}{,}\PY{l+s+s1}{\PYZsq{}}\PY{p}{)}\PY{p}{)}\PY{p}{;}
                \PY{n}{C\PYZus{}GVals}\PY{l+m+mf}{.}\PY{n}{field} \PY{o}{:=} \PY{n}{string\PYZus{}to\PYZus{}array}\PY{p}{(}\PY{n}{NEW}\PY{l+m+mf}{.}\PY{n}{c\PYZus{}g}\PY{p}{,} \PY{l+s+s1}{\PYZsq{}}\PY{l+s+s1}{,}\PY{l+s+s1}{\PYZsq{}}\PY{p}{)}\PY{p}{;}
        
                \PY{n}{maxLength} \PY{o}{:=} \PY{n}{getMaxLength}\PY{p}{(}
                        \PY{n}{ReferenceCindocVals}\PY{p}{,}
                        \PY{n}{DiscriminantVals}\PY{p}{,}
                        \PY{n}{VilleVals}\PY{p}{,}
                        \PY{n}{SujetVals}\PY{p}{,}
                        \PY{n}{NoteBPVals}\PY{p}{,}
                        \PY{n}{FicNumVals}\PY{p}{,}
                        \PY{n}{Idx\PYZus{}IcoVals}\PY{p}{,}
                        \PY{n}{NbrCliVals}\PY{p}{,}
                        \PY{n}{TailleCliVals}\PY{p}{,}
                        \PY{n}{N\PYZus{}VVals}\PY{p}{,}
                        \PY{n}{C\PYZus{}GVals}
                \PY{p}{)}\PY{p}{;}
        
                \PY{c+c1}{\PYZhy{}\PYZhy{}\PYZhy{}\PYZhy{}\PYZhy{}\PYZhy{}\PYZhy{}\PYZhy{}\PYZhy{}\PYZhy{}\PYZhy{}\PYZhy{}\PYZhy{}\PYZhy{}\PYZhy{}\PYZhy{}\PYZhy{}\PYZhy{}\PYZhy{}\PYZhy{}\PYZhy{}\PYZhy{}\PYZhy{}\PYZhy{}\PYZhy{}\PYZhy{}\PYZhy{}\PYZhy{}\PYZhy{}\PYZhy{}\PYZhy{}\PYZhy{}\PYZhy{}\PYZhy{}\PYZhy{}\PYZhy{}\PYZhy{}\PYZhy{}\PYZhy{}\PYZhy{}\PYZhy{}\PYZhy{}\PYZhy{}\PYZhy{}\PYZhy{}\PYZhy{}\PYZhy{}\PYZhy{}\PYZhy{}\PYZhy{}\PYZhy{}\PYZhy{}\PYZhy{}\PYZhy{}\PYZhy{}\PYZhy{}\PYZhy{}\PYZhy{}\PYZhy{}\PYZhy{}\PYZhy{}\PYZhy{}\PYZhy{}\PYZhy{}\PYZhy{}\PYZhy{}\PYZhy{}\PYZhy{}\PYZhy{}\PYZhy{}\PYZhy{}\PYZhy{}\PYZhy{}\PYZhy{}\PYZhy{}\PYZhy{}\PYZhy{}}
                \PY{c+c1}{\PYZhy{}\PYZhy{} Insérer dans la bonne table les données séparées}
                \PY{c+c1}{\PYZhy{}\PYZhy{}\PYZhy{}\PYZhy{}\PYZhy{}\PYZhy{}\PYZhy{}\PYZhy{}\PYZhy{}\PYZhy{}\PYZhy{}\PYZhy{}\PYZhy{}\PYZhy{}\PYZhy{}\PYZhy{}\PYZhy{}\PYZhy{}\PYZhy{}\PYZhy{}\PYZhy{}\PYZhy{}\PYZhy{}\PYZhy{}\PYZhy{}\PYZhy{}\PYZhy{}\PYZhy{}\PYZhy{}\PYZhy{}\PYZhy{}\PYZhy{}\PYZhy{}\PYZhy{}\PYZhy{}\PYZhy{}\PYZhy{}\PYZhy{}\PYZhy{}\PYZhy{}\PYZhy{}\PYZhy{}\PYZhy{}\PYZhy{}\PYZhy{}\PYZhy{}\PYZhy{}\PYZhy{}\PYZhy{}\PYZhy{}\PYZhy{}\PYZhy{}\PYZhy{}\PYZhy{}\PYZhy{}\PYZhy{}\PYZhy{}\PYZhy{}\PYZhy{}\PYZhy{}\PYZhy{}\PYZhy{}\PYZhy{}\PYZhy{}\PYZhy{}\PYZhy{}\PYZhy{}\PYZhy{}\PYZhy{}\PYZhy{}\PYZhy{}\PYZhy{}\PYZhy{}\PYZhy{}\PYZhy{}\PYZhy{}\PYZhy{}}
                \PY{k}{INSERT} \PY{k}{INTO} \PY{n}{Correction} \PY{p}{(}\PY{n}{ReferenceCindoc}\PY{p}{,} \PY{n}{Serie}\PY{p}{,} \PY{n}{Article}\PY{p}{,} \PY{n}{Discriminant}\PY{p}{,} \PY{n}{Ville}\PY{p}{,}
                        \PY{n}{Sujet}\PY{p}{,} \PY{n}{DescDet}\PY{p}{,} \PY{n+nb}{Date}\PY{p}{,} \PY{n}{NoteBP}\PY{p}{,} \PY{n}{FicNum}\PY{p}{,} \PY{n}{Idx\PYZus{}Ico}\PY{p}{,} \PY{n}{NbrCli}\PY{p}{,}
                        \PY{n}{TailleCli}\PY{p}{,} \PY{n}{Taille}\PY{p}{,} \PY{n}{N\PYZus{}V}\PY{p}{,} \PY{n}{C\PYZus{}G}\PY{p}{,} \PY{n}{Remarques}\PY{p}{)}
                    \PY{k}{VALUES}\PY{p}{(}
                        \PY{n}{unnest}\PY{p}{(}\PY{n}{array\PYZus{}expand}\PY{p}{(}\PY{n}{ReferenceCindocVals}\PY{l+m+mf}{.}\PY{n}{field}\PY{p}{,} \PY{n}{maxLength}\PY{p}{,}
        \PY{n}{ReferenceCindocVals}\PY{l+m+mf}{.}\PY{n}{field}\PY{p}{[}\PY{l+m+mf}{1}\PY{p}{]}\PY{p}{)}\PY{p}{)}\PY{p}{,}
                        \PY{n}{NEW}\PY{l+m+mf}{.}\PY{n}{serie}\PY{p}{,}
                        \PY{k}{cast}\PY{p}{(}\PY{n}{NEW}\PY{l+m+mf}{.}\PY{n}{article} \PY{k}{as} \PY{n+nb}{int}\PY{p}{)}\PY{p}{,}
                        \PY{n}{unnest}\PY{p}{(}\PY{n}{array\PYZus{}expand}\PY{p}{(}\PY{n}{DiscriminantVals}\PY{l+m+mf}{.}\PY{n}{field}\PY{p}{,} \PY{n}{maxLength}\PY{p}{,}
        \PY{n}{DiscriminantVals}\PY{l+m+mf}{.}\PY{n}{field}\PY{p}{[}\PY{l+m+mf}{1}\PY{p}{]}\PY{p}{)}\PY{p}{)}\PY{p}{,}
                        \PY{n}{unnest}\PY{p}{(}\PY{n}{array\PYZus{}expand}\PY{p}{(}\PY{n}{VilleVals}\PY{l+m+mf}{.}\PY{n}{field}\PY{p}{,} \PY{n}{maxLength}\PY{p}{,} \PY{n}{VilleVals}\PY{l+m+mf}{.}\PY{n}{field}\PY{p}{[}\PY{l+m+mf}{1}\PY{p}{]}\PY{p}{)}\PY{p}{)}\PY{p}{,}
                        \PY{n}{unnest}\PY{p}{(}\PY{n}{array\PYZus{}expand}\PY{p}{(}\PY{n}{SujetVals}\PY{l+m+mf}{.}\PY{n}{field}\PY{p}{,} \PY{n}{maxLength}\PY{p}{,} \PY{n}{SujetVals}\PY{l+m+mf}{.}\PY{n}{field}\PY{p}{[}\PY{l+m+mf}{1}\PY{p}{]}\PY{p}{)}\PY{p}{)}\PY{p}{,}
                        \PY{n}{NEW}\PY{l+m+mf}{.}\PY{n}{DescDet}\PY{p}{,}
                        \PY{n}{traitement\PYZus{}date}\PY{p}{(}\PY{n}{NEW}\PY{l+m+mf}{.}\PY{n+nb}{DATE}\PY{p}{)}\PY{p}{,}
                        \PY{n}{unnest}\PY{p}{(}\PY{n}{array\PYZus{}expand}\PY{p}{(}\PY{n}{NoteBPVals}\PY{l+m+mf}{.}\PY{n}{field}\PY{p}{,} \PY{n}{maxLength}\PY{p}{,} \PY{n}{NoteBPVals}\PY{l+m+mf}{.}\PY{n}{field}\PY{p}{[}\PY{l+m+mf}{1}\PY{p}{]}\PY{p}{)}\PY{p}{)}\PY{p}{,}
                        \PY{n}{unnest}\PY{p}{(}\PY{n}{array\PYZus{}expand}\PY{p}{(}\PY{n}{FicNumVals}\PY{l+m+mf}{.}\PY{n}{field}\PY{p}{,} \PY{n}{maxLength}\PY{p}{,} \PY{n}{FicNumVals}\PY{l+m+mf}{.}\PY{n}{field}\PY{p}{[}\PY{l+m+mf}{1}\PY{p}{]}\PY{p}{)}\PY{p}{)}\PY{p}{,}
                        \PY{n}{unnest}\PY{p}{(}\PY{n}{array\PYZus{}expand}\PY{p}{(}\PY{n}{Idx\PYZus{}IcoVals}\PY{l+m+mf}{.}\PY{n}{field}\PY{p}{,} \PY{n}{maxLength}\PY{p}{,}
        \PY{n}{Idx\PYZus{}IcoVals}\PY{l+m+mf}{.}\PY{n}{field}\PY{p}{[}\PY{l+m+mf}{1}\PY{p}{]}\PY{p}{)}\PY{p}{)}\PY{p}{,}
                        \PY{n}{unnest}\PY{p}{(}\PY{n}{array\PYZus{}expand}\PY{p}{(}\PY{n}{NbrCliVals}\PY{l+m+mf}{.}\PY{n}{field}\PY{p}{,} \PY{n}{maxLength}\PY{p}{,} \PY{n}{NbrCliVals}\PY{l+m+mf}{.}\PY{n}{field}\PY{p}{[}\PY{l+m+mf}{1}\PY{p}{]}\PY{p}{)}\PY{p}{)}\PY{p}{,}
                        \PY{n}{unnest}\PY{p}{(}\PY{n}{array\PYZus{}expand}\PY{p}{(}\PY{n}{TailleCliVals}\PY{l+m+mf}{.}\PY{n}{field}\PY{p}{,} \PY{n}{maxLength}\PY{p}{,}
        \PY{n}{TailleCliVals}\PY{l+m+mf}{.}\PY{n}{field}\PY{p}{[}\PY{l+m+mf}{1}\PY{p}{]}\PY{p}{)}\PY{p}{)}\PY{p}{,}
                        \PY{n}{NEW}\PY{l+m+mf}{.}\PY{n}{Taille}\PY{p}{,}
                        \PY{n}{unnest}\PY{p}{(}\PY{n}{array\PYZus{}expand}\PY{p}{(}\PY{n}{N\PYZus{}VVals}\PY{l+m+mf}{.}\PY{n}{field}\PY{p}{,} \PY{n}{maxLength}\PY{p}{,} \PY{n}{N\PYZus{}VVals}\PY{l+m+mf}{.}\PY{n}{field}\PY{p}{[}\PY{l+m+mf}{1}\PY{p}{]}\PY{p}{)}\PY{p}{)}\PY{p}{,}
                        \PY{n}{unnest}\PY{p}{(}\PY{n}{array\PYZus{}expand}\PY{p}{(}\PY{n}{C\PYZus{}GVals}\PY{l+m+mf}{.}\PY{n}{field}\PY{p}{,} \PY{n}{maxLength}\PY{p}{,} \PY{n}{C\PYZus{}GVals}\PY{l+m+mf}{.}\PY{n}{field}\PY{p}{[}\PY{l+m+mf}{1}\PY{p}{]}\PY{p}{)}\PY{p}{)}\PY{p}{,}
                        \PY{n}{NEW}\PY{l+m+mf}{.}\PY{n}{Remarques}
                    \PY{p}{)}\PY{p}{;}
        
                \PY{k}{RETURN} \PY{n}{NEW}\PY{p}{;}
            \PY{k}{END}\PY{p}{;}
        \PY{l+s}{\PYZdl{}}\PY{l+s}{\PYZdl{}} \PY{k}{LANGUAGE} \PY{n}{plpgsql}\PY{p}{;}
        
        \PY{c+cm}{/*}\PY{c+cm}{*}
        \PY{c+cm}{ }\PY{c+cm}{*}\PY{c+cm}{ TRIGGER: trigger\PYZus{}ndd\PYZus{}003}
        \PY{c+cm}{ }\PY{c+cm}{*}\PY{c+cm}{ Trello ref: NDD\PYZhy{}003: Séparer les lignes combinées}
        \PY{c+cm}{ }\PY{c+cm}{*/}
        \PY{k}{DROP} \PY{k}{TRIGGER} \PY{k}{IF} \PY{k}{EXISTS} \PY{n}{trigger\PYZus{}ndd\PYZus{}003} \PY{k}{ON} \PY{n}{CorrectionTemp}\PY{p}{;}
        \PY{k}{CREATE} \PY{k}{TRIGGER} \PY{n}{trigger\PYZus{}ndd\PYZus{}003}
            \PY{k}{BEFORE} \PY{k}{INSERT} \PY{k}{ON} \PY{n}{CorrectionTemp}
            \PY{k}{FOR} \PY{k}{EACH} \PY{k}{ROW}
                \PY{k}{EXECUTE} \PY{k}{PROCEDURE} \PY{n}{separer\PYZus{}ligne\PYZus{}combinees}\PY{p}{(}\PY{p}{)}\PY{p}{;}
\end{Verbatim}

    \begin{Verbatim}[commandchars=\\\{\},fontsize=\footnotesize]
NOTICE:  trigger "trigger\_ndd\_003" for relation "correctiontemp" does not exist,
skipping

    \end{Verbatim}

   \begin{Verbatim}[commandchars=\\\{\},fontsize=\scriptsize]
{\color{incolor}In [{\color{incolor}5}]:} \PY{k}{COPY} \PY{n}{CorrectionTemp}\PY{p}{(} \PY{n}{ReferenceCindoc}\PY{p}{,} \PY{n}{Serie}\PY{p}{,} \PY{n}{Article}\PY{p}{,} \PY{n}{Discriminant}\PY{p}{,} \PY{n}{Ville}\PY{p}{,} \PY{n}{Sujet}\PY{p}{,}
        \PY{n}{DescDet}\PY{p}{,} \PY{n+nb}{Date}\PY{p}{,} \PY{n}{Notebp}\PY{p}{,} \PY{n}{FicNum}\PY{p}{,} \PY{n}{Idx\PYZus{}Ico}\PY{p}{,} \PY{n}{NbrCli}\PY{p}{,} \PY{n}{TailleCli}\PY{p}{,} \PY{n}{Taille}\PY{p}{,} \PY{n}{N\PYZus{}V}\PY{p}{,} \PY{n}{C\PYZus{}G}\PY{p}{,} \PY{n}{Remarques} \PY{p}{)}
        \PY{k}{FROM} \PY{l+s+s1}{\PYZsq{}}\PY{l+s+s1}{/home/minhhoangdang/L3/S5/BD/TEA/pdfsrc/data.csv}\PY{l+s+s1}{\PYZsq{}} \PY{k}{DELIMITER} \PY{l+s+s1}{\PYZsq{}}\PY{l+s+s1}{     }\PY{l+s+s1}{\PYZsq{}} \PY{k}{CSV} \PY{k}{HEADER}\PY{p}{;}
        \PY{c+c1}{\PYZhy{}\PYZhy{}DROP TABLE IF EXISTS CorrectionTemp;}
        \PY{c+c1}{\PYZhy{}\PYZhy{}SELECT * FROM Correction LIMIT 10;}
\end{Verbatim}

   \begin{Verbatim}[commandchars=\\\{\},fontsize=\scriptsize]
{\color{incolor}In [{\color{incolor}6}]:} \PY{k}{COPY} \PY{n}{Correction} \PY{k}{TO} \PY{l+s+s1}{\PYZsq{}}\PY{l+s+s1}{/home/minhhoangdang/L3/S5/BD/TEA/pdfsrc/separated\PYZus{}data.csv}\PY{l+s+s1}{\PYZsq{}}
        \PY{k}{DELIMITER} \PY{l+s+s1}{\PYZsq{}}\PY{l+s+s1}{     }\PY{l+s+s1}{\PYZsq{}} \PY{k}{CSV} \PY{k}{HEADER}\PY{p}{;}
\end{Verbatim}
\end{comment}
    \hypertarget{les-cantons-associuxe9s-aux-communes}{%
\subsection{Les Cantons associés aux
communes}\label{les-cantons-associuxe9s-aux-communes}}

Après avoir récupéré un fichier csv d'une opendata contenant les
coordonnées lambert 93 (lambertX,lambertY) de toutes les villes de
france. - on mets à jour notre table pour ajouter deux colonnes LambertX
et LambertY. - on crée une nouvelle table Villes qui contiendra les noms
ainsi que les coordonnées Lambert de chaque ville. - on update notre
table pour insérer les coordonnées.

\hypertarget{conclusion}{%
\subsection{Conclusion:}\label{conclusion}}

Pour résumer, le prétraitement se divise en sous-tâches, exécutées dans
l'ordre suivant: \\
$\>$- Créer deux tables: \\
$\>$- Une table pour accueillir les données brutes \\
$\>$- Une table pour accueillir les données traitées
$\>$- Séparer les lignes combinées:\\
$\>$- Éliminer des incohérences (regexp + fonctions)\\
$\>$- Transférer les données dans la bonne table (INSERT)\\ 
$\>$- Compléter les informations - Suppression des informations insignifiantes\\
$\>$- Ajouter les coordonnées\\

   \begin{Verbatim}[commandchars=\\\{\},fontsize=\scriptsize]
{\color{incolor}In [{\color{incolor} }]:} 
\end{Verbatim}


    % Add a bibliography block to the postdoc
    
    
    
    \end{document}
